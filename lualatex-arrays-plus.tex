\documentclass[12pt]{article}

\usepackage[english,french]{babel}

\newcommand{\infilesd}[1]{files/#1}
\directlua{dofile(kpse.find_file("\infilesd{split-plus.lua}"))}
\newcommand{\luaarrayplus}[2][]{\directlua{mk_array_plus("#2","#1")}}

%%  Définition de 8 places au plus pour les arguments de la figure à tracer.
%%  On bourre avec des valeurs factices.
\newcommand{\getarrayplusone}{\empty}
\newcommand{\getarrayplustwo}{\empty}
\newcommand{\getarrayplusthree}{\empty}
\newcommand{\getarrayplusfour}{\empty}
\newcommand{\getarrayplusfive}{\empty}
\newcommand{\getarrayplussix}{\empty}
\newcommand{\getarrayplusseven}{\empty}
\newcommand{\getarraypluseight}{\empty}

\newcommand{\ifempty}[3]{{\edef\tmp{#1}\ifx\tmp\empty#2\else#3\fi}}

\newcommand{\makepicture}[8]{%  Dés qu'un argument est factice, on s'arrête.
 \ifempty{#1}{*}{%
  #1 --- \ifempty{#2}{*}{%
   #2 --- \ifempty{#3}{*}{%
    #3 --- \ifempty{#4}{*}{%
     #4 --- \ifempty{#5}{*}{%
      #5 --- \ifempty{#6}{*}{%
       #6 --- \ifempty{#7}{*}{#7 --- \ifempty{#8}{*}{#8}}}}}}}}}

\begin{document}

Début du tournage de la super-production.\bigskip

\begin{tabular}{*{6}{|c}|}
\hline
Titre 1 & Titre 2 & Titre 3 & Titre 4 & Titre 5 & Titre 6 \\ \hline
\luaarrayplus{\infilesd{arraydata.txt}}
\hline
\end{tabular}

\mbox{}

Il est impossible de définir --- ou de redéfinir --- des commandes tant que
nous sommes embourbés dans un environnement
\foreignlanguage{english}{\texttt{tabular}}. C'est pourquoi ces commandes sont
mises en réserve de la République par la fonction
\foreignlanguage{english}{\textsf{Lua} \texttt{mk\_array\_plus(\ldots,\ldots)}}
et c'est la fonction suivante
\foreignlanguage{english}{\texttt{put\_renewcommands()}} qui les envoie à
\foreignlanguage{english}{Lua\LaTeX} !

\directlua{put_renewcommands()}

Vous pouvez vérifier que les \emph{derniers} nombres de \emph{chaque} ligne
sont les arguments successifs de l'appel suivant. On suppose que le nombre de
lignes du fichier \foreignlanguage{english}{\textsf{.txt}} traité n'excède
pas~8.

\makepicture{%
 \getarrayplusone}{\getarrayplustwo}{\getarrayplusthree}{\getarrayplusfour}{%
 \getarrayplusfive}{\getarrayplussix}{\getarrayplusseven}{\getarraypluseight}

\end{document}
